Based on my previous experiences concerning the development tool for a third person, it was really important to me to define clearly the specification and limit of the project before starting to properly developp/ code it. This was on one hand to be sure to produce an interface that will suit my supervisors needs and on the other hand to avoid time waste in the future. Specifications concerning the graphical interface have been divided in five part: the basic requirements, the content, the design, the features and some further precisions. 

\newline \vspace{5mm}

During the early stage of the project we had several meetings with my two supervisor in order to define those specifications. These following specification are the specification we agreed at the time of writing the backgound and progress report.

\newline \vspace{5mm}


\begin{itemize}	
\item \textbf{Interface content:}
\newline \vspace{5mm}

\begin{itemize}
\item The interface should display patient images - images will be provided in DICOM format, and translated so that the patient can read them.
\item The interface must contain the clinical report and the simplifie version.
\item A link to NHS website will be given, so that patient could find general informations about their condition
\item Patient could get flag informations - to be filled by doctors - while exploring the images
\item Any other relevant informations related to what the DICOM files provides could be added 
\end{itemize}

\newline \vspace{5mm}

\item \textbf{Interface functionalities:}
\newline \vspace{5mm}
The interface should provide:
\begin{itemize}
\item One doctor oriented window: so, they can fill in datas (images, report) and add flag to images at their conveniance.
\item One patient oriented window: read only data (no modification allowed) and the possibility for patient to chat with doctors.

\end{itemize}

My main concern - in the context of this project - is to focus on the patient oriented side and see how far I can lead this project. This part can be really time consuming as it might need to be oftently readapted following the needs of my tutors.

Also, William recently sent me detailed general specifications concerning the patient oriented interface content/functions - Appendix 2

\newline \vspace{5mm}

\item \textbf{Interface design:}
\newline \vspace{5mm}
\begin{itemize}
\item Imaging display will depend on the provided images (MRI, CT) but not on the part of the body. Will also gave me on demand description concerning images to be display and the way to deal with it – Appendix 2.
\item Provide a side by side – or other relevant organization – that would allow the patient to get the images and the report together in a relevant way

\end{itemize}

\newline \vspace{5mm}

\item \textbf{Further precisions:}
\newline \vspace{5mm}
\begin{itemize}
\item No access to any database will be provided for the current project (security issues) 
\item Access to the interface will be local, patient would be given (upon request) a CD with their images loaded on the interface; this won’t change patient access to datas but should make them want to access it
\item Interface should include user specification/precisions for patient 
\item Benefits/specifications will have to be defined before starting implementation
\item Interface should be “windows portable” 
\end{itemize}

\newline \vspace{5mm}

\item \textbf{Interface evaluation:}
\newline \vspace{5mm}
At some point the interface should be evaluted by a panel of patient that would be ask to use it and make feedbacks.


\end{itemize}


\newline \vspace{5mm}
Eventually, thoses specifications have evolved during the realization of the project, adjusting the task the the work already performed and the encountered issues. Moreover concerning the interface Design, Will provided me with 



