Before we started to actually discuss the project goal, Will provided me with several documents, either literature reviews or articles already referenced above, concerning works and researches directly related to the project. Among those, he also provided me with the late stage report of his PhD, containing the state of play of his work at that stage. Those papers have allowed me to understand the background of the project and to identify the related issues and challenges. 

\newline \vspace{5mm}
The goal of Will's PhD is to identify the \textbf{risks and benefits} related to the creation of a patient-oriented interface. To be more precise I would cite directly from this report:

\newline \vspace{5mm}

\textit{``Intuitively, there are benefits available from sharing images with patients. Indeed, there is a wealth of research available which assesses how visualisation aids increased understanding, or promotes communication. However, little research assesses the value of radiological images in this context. Moreover, no work assesses the risks associated with sharing patients’ images with them. This is the gap which this PhD will address. The research questions for this PhD are, therefore:
\begin{itemize}
	\item Is there additional benefit that can be extracted from diagnostic images?
	\item If so, what are the requirements to enable the realisation of this benefit?"
\end{itemize}}

\newline \vspace{5mm}
Specifically, to lead his researche, he has divided benefits in two categories:
\textit{
\begin{itemize}
	\item ``Primary benefits – benefits related to the rationale for acquiring the image, e.g. diagnosis, assessment, interventional guidance
	\item Secondary benefits – benefits unrelated to the rationale for acquiring the image, e.g. education, communication, empowerment"
\end{itemize}
}

\newline \vspace{5mm}
At the moment, Will has achieved a significant work, he has completed a literature review, to set up the background and understand the issues related to the creation of a patient portal. Precisely, the aim was ``to ensure that this was a suitable subject for research and to identify any pertinent gaps in the existing knowledge.".  He also carried out a survey, questioning medical experts, supplemented with 8 semi-structured interviews in order identify and scale the related risks and benefits from a clinician point of view. the results of those works are available for consultation on \textbf{\textit{Appendix 1}. In the future the objective is to interview patients, on the same model as clinicians, to evaluate the non-clinician opinion.
  











