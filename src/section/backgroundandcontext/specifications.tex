--> Why define wpecification before beginning is important in that kind of project
\newline
--> Show specification defined with Will: can re use what was in background report


Based on my previous experiences concerning the development tool for a third person, it was really important to me to define clearly the specification and limit of the project before starting to properly developp/ code it. This was on one hand to be sure to produce an interface that will suit my supervisors needs and on the other hand to avoid time waste in the future. Specifications concerning the graphical interface have been divided in five part: the basic requirements, the content, the design, the features and some further precisions. 


During the early stage of the project we had several meetings with my two supervisor in order to define those specifications. Basically we agreed on the following one: 


\newline \vspace{5mm}
\textbf{1. Basics requirement}
= project goal


\newline \vspace{5mm}
\textbf{2. The content}


\newline \vspace{5mm}
\textbf{3. Features and functionalities}
The interface should provide:
\begin{itemize}
\item One doctor oriented window: so, they can fill in datas (images, report) and add flag to images at their conveniance.
\item One patient oriented window: read only data (no modification allowed) and the possibility for patient to chat with doctors.

\end{itemize}

My main concern - in the context of this project - is to focus on the patient oriented side and see how far I can lead this project. This part can be really time consuming as it might need to be oftently readapted following the needs of my tutors.

Also, William recently sent me detailed general specifications concerning the patient oriented interface content/functions - Appendix 1


\newline \vspace{5mm}
\textbf{4. The design}
In terms of design the first specifications were really basic, it was agreed that I should look around already existing imaging readers in order to get some ideas and that the creation will be done over time. 

\newline \vspace{5mm}
\textbf{5. Further Precision}
\begin{itemize}
\item No access to any database will be provided for the current project (security issues) 
\item Access to the interface will be local, patient would be given (upon request) a CD with their images loaded on the interface; this won’t change patient access to datas but should make them want to access it
\item Interface should include user specification/precisions for patient 
\item Benefits/specifications will have to be defined before starting implementation
\item Interface should be “windows portable” 
\end{itemize}


--> appendix: lately provided doc for design

