As mentioned in the introduction, \textbf{medical imaging} methods are widely used everyday for numerous reasons, including the diagnosis of more or less serious inner physical diseases. Using different imaging methods such as \textbf{Magnetic Resonance Imaging} (MRI), \textbf{Computerized Tomography} (CT-Scan) or \textbf{X-Ray}, this allow to produce a precise 2D or 3D representation of the internal structure of the human body. Since the introduction of the \textbf{DICOM standard} in 1985 - see following section - which provide a unified protocole to store and share imaging data, storage and communication has become easier between different healthcare services. However, when it comes to data sharing there is no uniformed protocol neither on the manner of doing so, nor on the fact that those data should be shared. Naturally, once an exam has been performed, patients will automatically be given their clinical results, in terms of wellness or sickness, but medical images won't be automatically shared with them. Depending on the issuing country, medical images can be shared, either automatically or on demand, more or less easily, with a patient.

\newline \vspace{5mm}

In the scope of this project, the targeted area is the \textbf{United Kingdom} where effective access to medical imaging data is currently not commonly provided. By default, once an exam is performed, only the medical report (a written interpretation of the imaging findings provided by a radiologist) will be send to the referenced general practitioner (GP) or other referring clinician. The patient will then need to make an appointment, with that clinician, in order to get the information concerning his medical condition. More precisely, he will get the clinical report communicated by his clinician. No medical images would necessarely be shared, or even shown at any time of the process, neither to the medical expert, nor to the patient. On demand, and provided payment of a defined amount, a patient can get his medical data on a CD or DVD, supplied with a default software reader built to deal with the special DICOM format. However, the kind of software that are provided on CDs or DVDs were initially made for clinician and may not be adapted for an average person. 


\newline \vspace{5mm}
Consequently, as patient desire to access their imaging data is growing, the idea to provide them with a suitable interface has become pretinent. By \textbf{suitable} is implied an interface that would only contains relevant and useful features, that would be instinctive, and contain information  that are reliable for the user. Indeed, it is more than easy nowadays to find a wealth of erroneous or misleading informations and explanations all over the Internet. The goal is to offer the patient a safe environment where he could explore his data and understand his medical condition without any drift. Such interfaces are commonly called \textbf{``patient portal"} and some have already been created and experimented locally on a sample of patient in Los Angeles [5]. However, this has only been part of an internal study and haven't reached the outside of it's location.

\newline \vspace{5mm}
Nevertheless, for the above reasons, building an interface for sharing medical diagnosis with imaging patients is not an trivial task. Medical images are sensitive pieces of information and a misunderstanding can be significant. What would then be the \textbf{risks and the benefits} of creating a patient portal? These questions have been subject to several literature reviews [1] and especially to the realisation of a \textbf{PhD} by one of my supervisors \textbf{William Cox}.
