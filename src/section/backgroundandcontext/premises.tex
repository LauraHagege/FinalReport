As mentionned in the introduction, \textbf{medical imaging} methods are widely used everyday to diagnose more or less serious intern physical diseases. Using different imaging methods such as \textbf{Magnetic Resonance Imaging} (MRI), \textbf{Computerized Tomography} (CT-Scan) or \textbf{X-Ray}, this allow to produce a precise 2D or 3D representation of the internal structure of the human body. Since the introduction of the \textbf{DICOM standard} in 1985 - developped therafter - which provide a unified protocole to store and share imaging data, storage and communication has become easier between different healthcare services. However, when it comes to data sharing there is no uniformed protocole neither on the manner nor on the fact that those data should be sahred. Naturally, once a exam has been perfomed, patients will automatically be given their clinical results, in terms of wellness or sickness but medical images will not compulsory be shared with them. Depending on the issuing country, medical images can be shared, either automatically or on demand, more or less easily with a patient.

\newline \vspace{5mm}

In the limit of this project, the targeted area is the \textbf{United Kindgom} where effectiv access to medical imaging data is not one of a main concern. By default, once an exam is performed, only the medical report will be send to the referenced general practician. The patient will then need to take an apointment in order to get the information concerning his medical condition, more precicely to get the clinical report told by his doctor. No medical images would be shared or even shown at anytime of the process,neither to the medical expert nor to the patient. On demand, and provided payment of a notable amount, a patient can get his medical data on a CD or DVD, supplied with a default software reader built to deal with the special DICOM format. However, the kind of software that are provided on CDs or DVDs were initially made for clinician and may not be adapted for an average person and this is where the project emerges. 


\newline \vspace{5mm}
While patient desire to access their imaging data is growing, the idea to provide them with a suitable interface have emerged. By \textbf{suitable} this mean an interface that would only contains relevant and useful features, that would be instinctive and contains informations  that are reliable for the user. Indeed, it is more than easy nowadays to find numerous of erroneous informations and explanations all over the Internet; the goal is to give the patient a safe environment where he could explore his data and understand his medical condition without any drift. Such interfaces are commonly called \textbf{"patient portal"} and the creation have already been some have already been created experimented locally on a sample of patient in Los Angeles [5]. However, this was, as my achieved work only part of a internal study and haven't reached the outside of it's location.

\newline \vspace{5mm}
Nevertheless, for numerous reasons, building an interface for sharing medical diagnosis with imaging patient is not an innocuous task. Medical imaging are sentitive pieces informations and a misunderstanding can be significant, we could then ask what would be the risk and the benefit of creating a patient portal? These question have been subject to several litterature reviews [1] and especially to the realisation of a PhD by one of my supervisor \textbf{William Cox}.
