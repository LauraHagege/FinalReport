Once familiar with the project background, it has been easier for me to understand the main \textbf{issues and challenges} and to have a precise idea of the goal to achieve. The objective of this project is not to simply copy and create another standard DICOM viewer but to create an interface that can be \textbf{progressively adapted to fit patient needs}. Challenges consist in buildging an interface that could be sufficient and valuable, meaning it could provide enough information to guide the patient on the right track without any simultaneous external contribution. Moreover, designing this interface should allow to identify the benefit to be realised through providing people with their imaging data, whilst also minimising the identified risks of this process.

\newline \vspace{5mm}
It has been agreed that we would together with Fernando and William, define project specifications (in terms of design and content) and that all information relative to benefits and content would be discussed mainly with William. They, also advised me to install some already existing DICOM viewer made for clinicians. They wanted me to get an idea of the kind of interface that I could produce, while keeping in mind that the goal is not to create a copy of those readers, but to produce my own patient-oriented interface.

\newline \vspace{5mm}
In the following part, I will develop the project specifications that have been produced in the early stage of the project, following multiple discussion with my tutors. Moreover, considering that I have been given complete freedom in the choice of the tools and languages to use for building the interface, I will explain the choices I have made before starting to implement my project.



