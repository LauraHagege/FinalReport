Once aware and familiar with the project background it was easier for me to understand the main issues and challenges and to have a precise idea of the goal to achieve. The objective of this project is not only to create a standard patient oriented interface but also to create an interface that can be \textbf{progressively adapted} in order to fit patient needs. On of the huge challenge is to create an interface that could be sufficiant and valuable, meaning it could provide enough information to guide the patient on the right track without any external contribution. 

\newline \vspace{5mm}
It has been agreed that we would together with Fernando and Will, define project specifications (in terms of design and content) and that all informations relative to benefits and content would be discussed mainly with Will. Fernando and William also advised me to install some already existing DICOM viewer made for clinicians in order the have an idea of the kind of interface that I could produce but while keeping in mind that the objectiv is not to simply do a copy of those readers.

\newline \vspace{5mm}
The following part I will developp the project specifications that have been realized in the early stage of the project. Moreover, given that I have been given complete freedom in the choice of the tools and languages to use for building the interface I will explain the choices I have made according to this.



