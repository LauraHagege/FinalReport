From the launch of the app through the effective display of the \textit{MainWindow}, the user will first have to come accross several steps/windows that are described below:

\begin{itemize}
	\item \textbf{Welcome page and Disclaimer}:
	
	\newline \vspace{3mm} 
	
	The first contact with the interface consist in a \textit{Welcoming Page} explaining the utility of the interface. User will get indication about the way of accessing his images and will be given a disclaimer, as follow:
	
	\newline \vspace{3mm}  
	
	``
	\textbf{Welcome} \\ 
	This interface is designed for patients to view their medical images. \\
	Please select the file corresponding to the study you wish to view (select folder SDY00XXX). \\
	Your images are protected by PIN.  \\
	Your pin is your date of birth in the following format: YYYY/MM/JJ
	
	\newline \vspace{3mm}  
	
	
	\textbf{Disclaimer:}
	\begin{itemize}
		\item These images are for information only 
		\item Do not try to interpret your images 
		\item For any queries, please contact your healthcare professional 
		\item Some people may find seeing images of themselves upsetting, please consider whether you want to view them 
		\item Be aware that these images constitute your personal medical data - please be mindful of who you share them with
	\end{itemize}''
	
	
	
	\newline \vspace{3mm}  
	
	An overview of this \textbf{Welcome Page} is available in \textbf{\textit{Appendix 9}}. This page is supported by a button \textit{Select File} to allow the user to choose the study he wish to display on the screen. By default the first image displayed will be the first image of the first serie.
	 
	 \clearpage
	
	\item \textbf{PIN Access}: 
	
	\newline \vspace{3mm} 
	
	As the interface is dealing with sentitive medical data, images won't show automatically once selected. Indeed, data are protected by a PIN access, which correspond, as described in the disclaimer, to the patient date of birth in the format YYYY/MM/JJ; an overview of this window is available on \textbf{\textit{Appendix 10 figure 18}}. If the PIN provided is not right, the user will be informed on his screen and have the right to write another code, see \textbf{\textit{Appendix 10, figure 19}}. If no PIN in provided and the window is closed by the user, the application will close automatically. However, if the user provide the right PIN, he will have access to his images and the Main Window will pop up.
	
	
	\item \textbf{Main Window}:
	
	\newline \vspace{3mm} 
	
	The \textit{MainWindow} is the heart of the interface, this is the window that will allow the user to view his images and select options. An overview of the screen is available on \textbf{\textit{Appendix 11}}, displaying the CT-scan of an abdomen. The architecture of this interface has been realized taking inspiration on the \textbr{Interface Design Specification} (Appendix 4) mentionned in part 3.1. All the available features concerning images treatment and information access will be described in the following part.
		
	
\end{itemize}