The first challenge of this project is to render a DICOM Image. As already explained the DICOM format is not trivial 

I don't know what would be written about dcmtk and DICOM 


Among other information I think this is relevant to explain how I managed to display the dicom images because this is not a trivial format

QT provide a class QIMage qnd blablabla

and had to get through dcmtk documentation to find the relevant function I will use !

\newline \vspace{5mm}

Given a DICOM folder, this should contain at least the DICOMDIR file and another folder call DICOMDAT containing paths to series folder and images.

Usually DICOM folder contains only on study composed of multiple series, each serie being stored in separate folders. Then one serie file can either contain:

\begin{itemize}
	\item One single frame DICOM Image
	\item Several single frame DICOM Images
	\item One multiframe DICOM Image
\end{itemize}

Objectively, single and multi frame images only differ by the size of the file.

\newline \vspace{5mm}	
\textbf{1. Display one image:}

\newline \vspace{5mm}	

The DCMTK library that I installed contains several classes that should make DICOM Images treatment easier.
The class I used to deal with DICOM Images files is called DicomImage, the class structure and related functions are available on DCMTK website [8888].
The class is provided with four different constructors and depending on the given parameters this class allows to deal with single frame and multiframe images. Information about the constructor I choosed are available on Appendix XXX figure YYY


\begin{itemize}
	\item Display single frame DICOM Image:
	
	\begin{center}
	\textit{DicomImage *DcmImg = new DicomImage(path)}
	\end{center}
	
	
	\item Display multiframe DICOM Image:
	
	\begin{center}
	\textit{DicomImage *DcmImg = new DicomImage(path,0,index,1)}
	\end{center}
	
	\textit{index} being the index of the frame to display
	
\end{itemize}

The variable \textit{path} is a string and contains the absolut path to the DICOM Image file.
\newline \vspace{5mm}


Once I got the DicomImage object, I need to get the pixels in order to have the opportunity to use QImage class thereafter. Here again DCMTK provide me with the function \textit{getOutputData()} - see Appendix XX figure ZZ -. The corresponding line of code is:
\newline \vspace{5mm} 

	\begin{center}
	\textit{uint8_t* pixelData= DcmImg->getOutputData(8)}
	\end{center}

\newline \vspace{5mm}  
Explain what is uint8_t



Finally I only need to use two classes provided by Qt to render the DICOM Image on my application:

\newline \vspace{5mm} 

	\begin{center}
	\textit{uint8_t* pixelData= DcmImg->getOutputData(8)}
	\end{center}

\newline \vspace{5mm}  

QIMage only takes pixel and a scene can only display QPixmap element see appendix


\newline \vspace{5mm}	

\textbf{2. Store and display successive images of the serie}

\newline \vspace{5mm}	




