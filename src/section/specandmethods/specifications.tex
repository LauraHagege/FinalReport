Based on my previous experiences concerning the development of a project for a third person, it was really important for me to clearly define the specifications and limits of the project, before starting the implementation. This was, on one hand, to be sure that I would produce an interface that would suit my supervisors needs, and on the other hand, to avoid time being wasted in the future. Specifications concerning the graphical interface have been divided into five parts: the basic requirements, the content, the design, the features and completed with further precisions. 

\newline \vspace{5mm}

During the early stage of the project we had several meetings with my two supervisors in order to define those specifications. These following specifications correspond to those agreed by the time of writing the \textbf{background and progress report}. Eventually, those specifications have evolved during the implementation of the project, adjusting tasks to the the achieved work and encountered issues. Precisely, concerning the design of the interface, specifications have evolved while discovering the interface designer. Further precision concerning the design of the interface has been provided by Will approaching mid July. He gave me a well detailed drawing of what he was expected supplement with more precisions concerning the features he was expecting. Those specifications were given as a reference to start the implementation of the interface architecture and are available on \textbf{\textit{Appendix 4}}.

\begin{itemize}	
\item \textbf{Basic requirement - main criteria of success:}
\begin{itemize}
\item Patient should be able to understand the provided images 
\item Patient could explore the data in different ways/ different images orientation
\item Patient should have the possibility to ask questions to doctors/ specific assigned people
\item At some point the interface should be evaluated by a panel of experts and laypersons that would be asked to use it and provide feedbacks.
\end{itemize}


\item \textbf{Interface content:}
\begin{itemize}
\item The interface should display patient images - images will be provided in DICOM format, and translated so that the patient can read them.
\item The interface must contain the clinical report and the simplified version.
\item A link to NHS website will be given, so that patient could find general information about their condition
\item Patient could get flag informations - to be filled by doctors - while exploring the images, for example, the image which best demonstrates any abnormality can be pulled out from the broader study dataset
\item Any other relevant information related to what the DICOM files provides could be added 
\end{itemize}


\item \textbf{Interface functionalities:}\\
The interface should provide:
\begin{itemize}
\item One doctor-oriented window: so, they can fill in the relevant data (images, report) and add a flag to appropriate images at their convenience.
\item One patient-oriented window: read only data (no modification allowed) and the possibility for patient to chat with doctors.

\end{itemize}

My main concern - in the context of this project - is to focus on the patient-oriented side and see how far I can lead this project. This part can be really time consuming as it might need to be frequently readapted following the needs of my tutors.

Also, Will gave me, back at the time, detailed general specifications concerning the patient-oriented interface content/functions that are available on \textbf{\textit{Appendix 2 and 3}}.


\item \textbf{Interface design:}
\begin{itemize}
\item Imaging display will depend on the provided images (MRI, CT) but not on the part of the body. Will also gave me on demand description concerning images to be display and the way to deal with it - Appendix XX -.
\item Provide a side by side – or other relevant organization – that would allow the patient to get the images and the report together in a relevant way
\end{itemize}


\item \textbf{Further precisions:}
\begin{itemize}
\item No access to any database will be provided for the current project (security and data protection issues) 
\item Access to the interface will be local, patient would be given (upon request) a CD with their images loaded on the interface; this won’t change patient access to data but should make them want to access meaningful
\item Interface should include user specifications/precisions for patient 
\item Benefits/specifications will have to be defined before starting implementation
\item Interface should be Microsoft Windows portable
\end{itemize}


\end{itemize}






