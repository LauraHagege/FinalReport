%background
In a world where access to data is becoming faster and easier, how could we prevent the society from the desire to have a better and more systematic \textbf{access to medical data}? In particular, in the field of \textbf{Medical Imaging}, which propose a range of routinely used methods, to produce precise representations of the inner human body and diagnose serious diseases? And especially, in the \textbf{United Kingdom}, where The sharing of medical images with patients is not routine practice? We can't, and some solutions are popping up on the way to deal with it.

\newline \vspace{5mm}
Indeed, some studies have already been undertaken in the United States concerning the creation of a \textbf{``patient portal"} [1]. Aim of these being to provide the patient with a \textbf{suitable graphical interface} that would display their medical images. \textbf{Prof. Fernando Bello} and \textbf{William Cox} sought to expand on the existing work in that particular area by assessing the benefits and risks that could come from the conception of such an interface. Indeed, they proposed to effectively implement this interface, as a final personal project to the \textbf{department of Computing Science at Imperial College of London} together with the \textbf{Chelsea and Westminster Hospital}. 

\newline \vspace{5mm}
Consequently, the following report offers an overview of the work undertaken during the last three months on the creation of \textbf{``A digital interface designed for sharing diagnostic medical imaging with patients"}. From specifications to conception, through design and testing, the aim of the following document is to present the different stages of the interface realisation, including issues, skills earned, and the final output that will be left as a basis for further development.





