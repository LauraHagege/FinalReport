%background
\textbf{Medical Imaging} is a generic term to mention medical techniques and processes that aim at exploring the inside of the human body. This expression gather together various clinical methods, such as, \textbf{Magnetic Resonance Imaging} (MRI), \textbf{Computerized Tomography} (CT-Scan) or \textbf{X-Ray}, which allow to produce a 2D or 3D representation of physical structures. Those imaging procedures seeks to reveal internal body structures and to diganose diseases in order to treat them. 

%focusing
\newline \vspace{5mm}

\textbf{Sharing data} with patient is an important part of the operation for medical imaging services -- like in every medical department.  With the evolution of technology, the procedure that was once systematically performed by a clinical expert, is now likely to be automised. However, provided patients have access to their data; the underlying challenge is to make the images understandable. Consequently, this automation  of data sharing should, as widely mentioned, come along with different \textbf{level of benefit}, meaning each provided information should be scaled according to it's benefit for the patient. 

%objectives
\newline \vspace{5mm}

Considering those facts, the project beeing presented in this report, has been proposed with the objectif of creating an \textbf{intuitive Graphical User Interface} (GUI) that would allow medical imaging patients to both \textbf{access and understand} their data in the easiest way. 

%methods
\newline \vspace{5mm}
To do so, together with \textbf{Dr Fernando Bello} and \textbf{Pr William Cox}, we first defined the global specification of the project in term of content, design and features. 
Then I have been left completely free and independant to choose the tools and devices to create the interface and to organize my work.
Step by step, ponctuated with regular meetings and feedback I have then been able to create the final ... 


%results
\newline \vspace{5mm}
The final interface consist in ... and the content, design and features are to be developp in this report.

\newline \vspace{5mm}
%conclusion
As some stuff haven't been achieved blablabla




