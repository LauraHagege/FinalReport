%background
In a world where access to data is becoming faster and easier everyday, how could we prevent the society from the desire to have a better and more systematic \textbf{access to medical data}? In particular, in the field of \textbf{Medical Imaging}, which propose a range of routinely used methods to produce precise representations of the inner humans body and diagnose serious diseases? And especially, in the \textbf{United Kingdom}, where medical imaging data sound like a well kept mysterious treasure? Apparently we can't, and some solutions are popping up on the way to deal with it.

\newline \vspace{5mm}
Indeed, some researches have already been led in the United States concerning the creation of a \textbf{"patient portal"} [1]. The main idea beeing to provide the patient with a suitable graphical interface that would display their medical images and help them to understand what they see. \textbf{Dr Fernando Bello} and \textbf{Pr William Cox} took the bet of digging in that particular area to explore the benefits and risks that could come from the conception of such an interface. They ended up to propose the effective realization of this interface, as a final personnal project to the \textbf{departement of Computing Science at Imperial College of London} together with the \textbf{Chelsea and Westminster Hospital}. 

\newline \vspace{5mm}
Consequently, the following report offers an overview of the work undertaken during the last three month on the creation of \textbf{"A digital interface designed for sharing diagnostic medical imaging with patients"}. From specifications to conception, throught design and testing, the aim of the following document is to present the different stage of the interface realisation, including issues, skills earned, and the final output that will be left as a basis for further development.





