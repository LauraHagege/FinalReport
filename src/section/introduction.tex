According to a study published in January 2018 from \textbf{National Health Service (NHS) England} [2], 41.4 million of \textbf{imaging tests} have been reported in England between October 2016 and December 2017. Indeed, \textbf{medical Imaging} exams are routinely used all over the world to explore internal body structures and/or diagnose diseases. This generic term gather together various clinical methods, such as, \textbf{Magnetic Resonance Imaging} (MRI), \textbf{Computerized Tomography} (CT-Scan) or \textbf{X-Ray}, which allow to produce a 2D or 3D representation of physical intern structures. 

\newline \vspace{5mm}

Methods used for those review are common, and since the introduction of the \textbf{DICOM standard} in 1985 [3], so is the professional storage and communication system. However, when it comes to sharing diagnoses with patient, each country get to developp its own methods. In the \textbf{United Kingdom}, it is less than likely that a patients get, or even ask, access to their medical data. On demand and providing payment, one can get his plain clinical images. But generally, patients have only access to their clinical report, by means of a general practitioner, and the medical images remain a mystery black box that no one get to see. 

\newline \vspace{5mm}

Simultaneously, with the evolution of technology making access to knowledge easier day after day, access to \textbf{medical informations} is also more coveted and especially concerning \textbf{imaging data}. However, the issue about sharing those sensitive data, is not only to promote access, but essentially to make those images affordable for the common mortal. Researches have already been made in the United States concerning the creation of a \textbf{"patient portal"} [4] - fully designed to facilitate \textbf{patient understanding} - exploring the related opportunities, and scaling different levels of benefit. 

\newline \vspace{5mm}

Taking all those facts into consideration, \textbf{Dr Fernando Bello} and \textbf{Pr William Cox} have decided to deepen the subject of designing a \textbf{patient portal}. First by exploring the \textbf{benefits and risks} from sharing medical informations with patient and thereafter by effectively building such an interface. Consequently, after a year of background researches, they offered to work on the creation of \textbf{"A digital interface designed for sharing diagnostic medical imaging with patients"}, as a final personnal project to the \textbf{departement of Computing Science at Imperial College of London}, together with the \textbf{Chelsea and Westminster Hospital}. Aiming to build an interface \textbf{suitably made for imaging patients} in order to let them access and understand their medical results in the most significant and comprehensible way.

\newline \vspace{5mm}
The following report contains an overview of the work undertaken during almost three month beginning late May 2018 under the supervision of \textbf{Dr Fernando Bello} and \textbf{Pr William Cox}. The following parts will developp in a logical order - for better understanding - the several steps reached during the conception of the interface, including issues, skills earned and personnal review. 

\newline \vspace{5mm}
//Introduce the plan - will do once it is certain

  


 





