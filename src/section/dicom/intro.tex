\textbf{DICOM standard} is a special software integrated format dedicated to \textbf{ease data storage and communication} between different facilities in the \textbf{medical imaging field}. This standard has been defined by the \textbf{American College of Radiology} (ACR) and the \textbf{National Electrical Manufactural Association} (NEMA) in 1985. DICOM defines a specific data model structure, a file format and data dictionary, it also comes with a TCP/IP protocol to facilitate data transfer. Before the creation of this standard, it was challenging for distinct services to exchange imaging information, currently DICOM format is widely use among the medical imaging area. 

\newline \vspace{5mm}
DICOM standard is a well-structured but also hard to reach software. Even if its website [7] provides a wide documentation, it remains challenging to fully understand it. The aim of the following subsection is to give a basic overview of DICOM standard structures. In the scope of my project I will mainly focus on the data storage and treatment, TCP/IP protocol has not been part of my concern. Moreover, as complete the DICOM standard website could be, I used several websites [8] \& [9], to build my understanding on the DICOM file format. 



