\textbf{1.Definition}

\newline \vspace{5mm}
DICOM file format is a special software integrated standard format dedicated to ease
data communication between different facilities in the Medical Imaging Field. This standard has been defined by the American College of Radiology (ACR) and the National Electrical Manifactural Association (NEMA) in 1985. DICOM format defines among a lot of others data dictionnary, data structure, file format and comes with a TCP/IP protocole to facilitate data transfer. Before the creation of this standard, it was difficult for different facilities to exchange imaging and imformations, currently this format is widely use for all medical imaging areas such as CT (Computed Tomography), MRI (Magnetic Reasonance Imaging), X-Rays, Ultrasounds, etc. 

\newline \vspace{5mm}
DICOM format is a well structured piece of work and DICOM standard website [1] provides a large documentation, however, it remains a challenge to fully understand what this is all about.My objective in the following subsection is to give a basic overview of DICOM standard features. As complete the DICOM standard could be, I also used two well formed websites [2] [3], to build my understanding on the DICOM file format. 

\newline \vspace{5mm}



